\chapter{Progettazione}

Progettare per:
\begin{itemize}
\item Dominare la complessità del prodotto ('divide-et-impera')
\item Organizzare e ripartire le responsabilità di realizzazione
\item Produrre in economia (efficienza)
\item Garantire qualità (efficacia)
\end{itemize}


L'analisi usa un approccio investigativo per capire quale è il problema.
La progettazione usa un approccio sintetico per descrivere come arrivare alla soluzione al meglio.

Problema $\to$ l'analisi approfondisce il problema $\to$ Requisiti $\to$ la progettazione sintetizza la soluzione $\to$ Soluzione

\section{La modellazione}

La modellazione è il processo che sviluppa modelli astratti di un sistema, dove ogni modello rappresenta una diversa prospettiva del sistema.
Un modello non è la rappresentazione completa del sistema, in quanto tralascia volontariamente dei dettagli per rendere più comprensibile il sistema, a meno che i modelli non vengano usati direttamente per l'implementazione, in quel caso devono essere dettagliati e completi.
Ci sono diverse prospettive secondo cui modellare:
\begin{itemize}
\item prospettiva esterna, viene modellato il contesto in cui opera il sistema
\item prospettiva di interazioni, vengono modellate le interazioni sistema-ambiente o tra i componenti del sistema
\item prospettiva strutturale, viene modellata la struttura del sistema o l'organizzazione dei dati elaborati dal sistema
\item prospettiva comportamentale, viene modellato il comportamento dinamico del sistema
\end{itemize}

%%%%%% WIP %%%%%%%%
%\subsection{Modelli contestuali}


%\subsection{Modelli di interazione}

%Diagrammi dei casi d'uso e di sequenza

%\subsection{Modelli strutturali}

%Diagrammi di classe 

%\subsection{Modelli comportamentali}

