\chapter{Design Pattern}
\section{Design Pattern, cosa sono?}
I \textbf{Design Pattern} sono \textit{soluzioni progettuali generali a problemi ricorrenti}. Si tratta di una descrizione o modello logico da applicare per risolvere un problema, il quale può presentarsi in diverse situazioni durante le fasi di progettazione e sviluppo del software.\\
Vanno definiti prima della codifica, questo permette di contenere e ridurre i problemi futuri dello sviluppo di un software.\\
I design pattern tipicamente mostrano relazioni ed interazioni tra \textit{classi} o \textit{oggetti}, senza specificare le classi applicative finali coinvolte.

\subsection{Come sono costituiti}
Un \textit{design pattern} è costituito da:
\begin{itemize}
	\item \textit{il nome}, costituito da una o più parole che siano rappresentative del pattern stesso;
	\item \textit{il problema}, ovvero la descrizione della situazione alla quale si può applicare il pattern. Può comprendere la descrizione di classi o di problemi di progettazione specifici, come anche una lista di condizioni perchè sia necessario l'utilizzo del pattern
	\item \textit{la soluzione}, che descrive gli elementi che costituiscono il progetto, le relazioni e le relative implicazioni, senza addentrarsi in una specifica implementazione. In poche parole bisogna rappresentare un problema astratto e la relativa configurazione di elementi adatta a risolverlo.
	\item \textit{le conseguenze}, risultati e vincoli che derivano dall'applicazione del pattern. Le conseguenze comprendono considerazioni di tempo e di spazio, possono descrivere implicazioni del pattern con alcuni linguaggi di programmazione e l'impatto con il resto del progetto. Sono fondamentali in quanto aiutano nella scelta dei pattern.
\end{itemize}
I design pattern possono essere classificati in tre principali categorie, che risolvono problemi diversi. Questi tipi di pattern sono di tipo \textit{Strutturale, Creazionale e Comportamentale}, e verranno approfonditi nei prossimi paragrafi.

\section{Design Pattern Strutturali}
I \textit{design pattern strutturali} sono relativi a come classi ed oggetti sono composti per formare strutture più complesse. In generale possiamo distinguere due tipologie di pattern strutturali:
\begin{itemize}
\item basati su \textbf{classi}, i quali utilizzano l'ereditarietà per generare classi che combinano le proprietà di classi base.
\item basati su \textbf{oggetti}, i quali mostrano come comporre gli oggetti al fine di estendere, in fase di esecuzione, le funzionalità di una classe (cosa non possibile nel caso della composizione statica tramite ereditarietà).
\end{itemize}
La maggior parte dei pattern strutturali sono basati sugli oggetti.
Di seguito andremo ad analizzare i principali pattern strutturali conosciuti.
\subsection{Adapter}
\subsection{Bridge}
\subsection{Composite}
\subsection{Container}
\subsection{Decorator}
\subsection{Façade}
\subsection{Proxy}
\subsection{Flyweight}

\section{Design Pattern Creazionali}

\section{Design Pattern Comportamentali}

